\documentclass[12pt]{article}

\begin{document}

\clearpage

\section*{ASSIGNMENT-6}

\subsection*{}
\begin{itemize}
\item Name       :  Medha Kumari
\item Roll no.   :  2111030
\end{itemize}

\section*{SOLUTIONS PROVIDED BY BIOMEDICAL ENGINEERS DURING COVID-19}
Due of the unique obstacles described by epidemiologists, immunologists, and medical practitioners, such as survival, symptoms, protein surface composition, and infection pathways, biomedical science and engineering have been proposed as promising areas to serve medical science in combating SARS-CoV-2.
\subsection*{Vaccine Development}
As bioengineered products and procedures are being investigated in practically every discipline of biomedical sciences, various lessons from domains like as cancer therapy and medication delivery can be applied to the development of better vaccine manufacturing technologies for respiratory disorders.
\subsection*{Diagnosis , Treatment and Prevention  }
A variety of medical equipment and devices are required for the diagnosis and treatment of Covid-19 disease, and the success of the pandemic response is linked to the availability of these medical equipment and gadgets.
\subsection*{Oxygen Therapy}
The delivery of additional oxygen using a nasal cannula or a more intrusive face mask is usually the primary form of treatment for mild respiratory insufficiency. The oxygen is usually delivered in cylinders, which are either tiny for transportation or big for fixed patients and longer-term supplies.

Although oxygen concentrators are an appealing option to tanks, they are rarely used in hospital settings for caring for COVID-19 patients. Oxygen concentrators take oxygen from the air and deliver it to the patient on demand. Concentrators are available in a variety of sizes, ranging from a small portable shoulder bag to larger fixed units for patients who require oxygen 24 hours a day.


High flow nasal oxygen (HFNO) is a type of oxygen delivery that delivers warmed and humidified oxygen at high flow rates (usually tens of litres/min) at body temperature and up to 100 percent RH and 100 percent oxygen to avoid drying out the airways.

\subsection*{Patient Monitorning}
One of the key parameters for COVID-19 patient is the amount of oxygen in their bloodstream (SpO2), measured by pulse oximetry which uses optics within a finger clamp. Pulse oximetry tends to be used for the duration of the patient’s stay in ICU.

Modern patient monitors provide many more patient parameters all the way to breathing waveforms to enable clinicians to fine tune their care of the patients.


The amount of oxygen in a patient's bloodstream (SpO2), which is evaluated by pulse oximetry, which uses optics within a finger clamp, is one of the most important metrics for COVID-19 patients. Pulse oximetry is often used for the duration of a patient's stay in the intensive care unit.

Modern patient monitors provide a plethora of additional patient parameters, all the way down to breathing waveforms, allowing doctors to fine-tune their patient treatment.

\subsection*{ Tracking and Testing }
When covid-19 pandemic occured we needed as many testing kits as possible to tackle the virus.It includes rapid tests which took short time to declare the results as compared to pcr tests.
Because of the risk of infection and quick transmission, the development of software and technical applications, such as tele-medicine to watch the virus's evolution in the population, has gotten a lot of interest.
Many testing kits were developed by the engineers.

\end{document}

\documentclass[12pt]{article}

\begin{document}

\clearpage

\section*{ASSIGNMENT-2}

\subsection*{}
\begin{itemize}
\item Name       :  Medha Kumari
\item Roll no.   :  2111030
\end{itemize}

\section*{Evolution of Modern Healthcare System}
\subsection*{From Beginning to Personalized Medicine}
The practice of medical care is as old as the first recordings in human history, conceiving the human body
and disease in a holistic manner. Early medical traditions include those of ancient Egypt and Babylon. The
Greeks went even further, introducing the concepts of
medical diagnosis, prognosis, and advanced medical
ethics. Around 800 BCE Homer in The Iliad gives descriptions of wound treatment by the two sons of
Asklepios, the admirable physicians Podaleirius and
Machaon and one acting doctor, Patroclus. The
Hippocratic Oath, still taken by doctors up to today,
was written in Greece in the fifth century BCE. In the
medieval age, surgical practices inherited from the ancient masters were improved and then systematized in
Rogerius’s The Practice of Surgery. Universities began
systematic training of physicians around the years 1220
in Italy. During the Renaissance, understanding of anatomy improved with pioneering works such as those
from Andreas Vesalius, and the microscope was
invented. The germ theory of disease in the nineteenth
century led to cures for many infectious diseases.
Military doctors advanced the methods of trauma treatment and surgery. Public health measures were developed especially in the nineteenth century as the rapid
growth of cities required systematic sanitary measures.
Advanced research centers opened in the early twentieth tury, often connected with major hospitals. The mid-twentieth century was characterized by new biological
treatments, such as antibiotics. These advancements,
along with developments in chemistry, genetics, and
lab technology (such as the x-ray) led to modern medicine. Medicine was heavily professionalized in the
twentieth century, and new careers opened to women
as nurses (from the 1870s) and as physicians (especially
after 1970) and at the same time disease specialization
arose in the sense of providing medical care with respect to the specific characteristics of the disease as
well as provided from specialized medical professionals.
The twenty-first century is characterized by highly advanced research involving numerous fields of science.
Yet, the advancement of medical care was moving
hand in hand with technological advancement. Besides
basic knowledge of biomedical processes, technology
was the main implementer of that knowledge. The
twenty-first century could be characterized by the phrase
Biology is the new Physics,stated in an article by
Philip Hunter in 2010.1 These very technological breakthroughs, which include the innovations in semi-conductors, the increase in computational power along with the lowering of cost of computational power, were the main
factors for the onset of personalized medicine. It would
be unthinkable to speak of such applications such as
providing health care based on the patients profile without the use of high throughput screening methods, such
as micro-arrays and next generation sequencing (NGS).
Or it would be impossible to interpret genomic data
without the analytical and mathematical tools from
physics, mathematics and engineering. Hence, it was
the marriage of biology and engineering that brought about such possibilities. Further on, pioneering work in
engineering has created new possibilities through the
application of bio-sensors, nano-particles and nano-bots,
advanced imaging methodologies, as well as improved
procedures towards the understanding of biological signal transduction and information transmission. This interdisciplinary interaction gave the possibility of predicting the health risks of an individual based on his/hers genomic profile (still considering that this process is in its infancy and there are more to be learned about). At the same time, those new analytical tools gave the possibility of treating human
disease from a holistic perspective, yet this time considering
biological systems as a complete entity and not as isolated
molecular events Hence, there should be an increasing effort towards the inter coupling of the biological sciences with engineering,
since this is, a one-way road to the improvement of medical
care and consequently to personalized medicine.

\end{document}

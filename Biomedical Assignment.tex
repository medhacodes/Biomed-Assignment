\documentclass[12pt]{article}
\usepackage{graphicx}
\graphicspath{{images}}
\usepackage{hyperref}
\hypersetup{colorlinks=true,citecolor=blue,linkcolor=blue}

\begin{document}
\tableofcontents


\section{X-Ray Machine}


n X-ray is a common imaging test that’s been used for decades. It can help your doctor view the inside of your body without having to make an incision. This can help them diagnose, monitor, and treat many medical conditions.

\begin{figure}
\centering
\includegraphics[scale=0.4]{x ray machine}
\caption{X-Ray Machine}
\end{figure}

\subsection{X-Ray parts}
X-Ray has 3 Main components
\begin{itemize}
\item operating console
\item High Frequency Generator
\item X-Ray Tube
\begin{itemize}
\item Internal
\item External
\end{itemize}
\item Other parts include
\begin{itemize}
\item collimator and grid
\item Bucky
\item X-Ray Film
\end{itemize}
\end{itemize}

\subsection{Working}

An X-ray is produced when a negatively charged electrode is heated by electricity and electrons are released, thereby producing energy. That energy is directed toward a metal plate, or anode, at high velocity and an X-ray is produced when the energy collides with the atoms in the metal plate.

\subsection{Uses}

\begin{itemize}
\item examine an area where you’re experiencing pain or discomfort
\item monitor the progression of a diagnosed disease, such as osteoporosis
\item check how well a prescribed treatment is working
\end{itemize}

\subsection{Benefits}

\begin{itemize}
\item noninvasively and painlessly help to diagnose disease and monitor therapy;
\item support medical and surgical treatment planning
\item guide medical personnel as they insert catheters, stents, or other devices inside the body, treat tumors, or remove blood clots or other blockages.
\end{itemize}

\subsection{Risks}

\begin{itemize}
\item a small increase in the possibility that a person exposed to X-rays will develop cancer later in life
\item tissue effects such as cataracts, skin reddening, and hair loss, which occur at relatively high levels of radiation exposure and are rare for many types of imaging exams
\end{itemize}

\section{MRI}
\begin{figure}
\centering
\includegraphics[scale=0.6]{Mri}
\caption{MRI machine}

\end{figure}
Magnetic resonance imaging (MRI) is a medical imaging technique that uses a magnetic field and computer-generated radio waves to create detailed images of the organs and tissues in your body. Most MRI machines are large, tube-shaped magnets

\subsection{MRI Parts}
\begin{itemize}
\item A tube like structure called 'Bore'
\item Magnet
\item Gradient coil
\item Shim coil,RF coil
\item Computer
\end{itemize}
\subsection{Working}
MRI makes use of the magnetic properties of certain atomic nuclei.Hydrogen nucleus(single proton)present in water molecules, and therefore in all body tissues.The Hydrogen  nuclei partially aligned by a strong magnetic field in the scanner.The nuclei can be rotated using radio waves, and they subsequently oscillate in magnetic field while returning to equilibrium.Simultaneously they emit a radio signal.This is detected using antennas (coils).Very detailed images can be made of soft tissues.

\subsection{Uses}
\begin{itemize}
\item anomalies of the brain and spinal cord
\item tumors, cysts, and other anomalies in various parts of the body
\item breast cancer screening for women who face a high risk of breast cancer
\item injuries or abnormalities of the joints, such as the back and knee
\item Certain types of heart problems
\item diseases of the liver and other abdominal organs
\item still expanding in many areas
\end{itemize}
\subsection{Benefits}
\begin{itemize}
\item No ionizing radiation
\item variable thickness in any plane
\item better contrastv solution
\item many details without iv contrast

\end{itemize}
\subsection{Risks}
\begin{itemize}


\item Very expensive
\item Dangerous For Patients with metallic devices placed within body
\item  difficult to be performed on claustrophic patients
\item during scanning may cause blurry images

\end{itemize}

\section{ECG MACHINE}
\begin{figure}
\centering
\includegraphics[scale=0.5]{Digital ECG machine}
\caption{ECG MACHINE}
\end{figure}

An electrocardiogram (ECG) is a simple test that can be used to check your heart's rhythm and electrical activity. Sensors attached to the skin are used to detect the electrical signals produced by your heart each time it beats.

\subsection{Parts}
\begin{itemize}
\item electrocardiograph
\item electrocardiogram
\item ecg paper
\item ecg leads

\end{itemize}

\subsection{Working}
The electrodes are connected to an ECG machine by lead wires. The electrical activity of the heart is then measured, interpreted, and printed out. No electricity is sent into the body. Natural electrical impulses coordinate contractions of the different parts of the heart to keep blood flowing the way it should.

\subsection{Uses}

\begin{itemize}
\item help diagnose and monitor conditions affecting the heart
\item It can be used to investigate symptoms of a possible heart problem, such as chest pain, palpitations (suddenly noticeable heartbeats), dizziness and shortness of breath.
\item Dizziness, lightheadedness or confusion
Heart palpitations
Rapid pulse
\end{itemize}]
 
 \subsection{Benefits}
 
\begin{itemize}
\item ECG is helpful to measure three basic parameters of clinical interest viz. rhythm and heart rate, axis of the heart and state of myocardial muscle.
\item ECG represents data in the topographic form which provides higher diagnostical information
\item ECG helps to prevent heart attacks by analyzing heart parametets at the initial stage.
\item ECG is used to detect the cardiac conditions of the patients after surgical or any other operation and after application of anesthesia.
\item ECG test is quick, painless and safe.

\end{itemize}
 \subsection{Risks}
 \begin{itemize}
 \item It does not provide underlying heart problems for patients not having any symptoms.
 \item It does not always provide help in accurate diagnosis. More tests are needed to trace serious heart problems undetected by normal ECG curve.
 
 \end{itemize}
 
\section{Ventilator}
\begin{figure}
\centering
\includegraphics[scale=0.8]{ventilator}
\caption{Ventilator}
\end{figure}


 A ventilator is a machine that provides mechanical ventilation by moving breathable air into and out of the lungs, to deliver breaths to a patient who is physically unable to breathe, or breathing insufficiently.
 \subsection{Ventilator Parts} 
 \begin{itemize}
 \item The Power Source
 \item The Controls
 \item The Monitors
 \item Safety Features

 \end{itemize}
 
 \subsection{Working}
 The machine works by bringing oxygen to the lungs and taking carbon dioxide out of the lungs. This allows a patient who has trouble breathing to receive the proper amount of oxygen. It also helps the patient's body to heal, since it eliminates the extra energy of labored breathing.
 
 \subsection{Uses}
 People require ventilation if they are experiencing respiratory failure. When this occurs, a person cannot get enough oxygen and may not be able to expel carbon dioxide very well either. It can be a life threatening condition

 There are many injuries and conditions that can cause respiratory failure, including
 \begin{itemize}
 \item head injury,stroke,lung disease
\item spinal cord injury
\item polio,sudden cardiac arrest,pneumonia
\end{itemize}

\subsection{Benefits}

\begin{itemize}

\item better gas distribution
\item lower meanway pressure
\item less hymodynamic disturbance
\item less sedation is required
\item weaning is easier(in most of the cases)


 \end{itemize}
 \subsection{Risks}
 \begin{itemize}
 
 
 \item The breathing tube in your airway could let in bacteria that infect the tiny air sacs in the walls of your lungs. Plus, the tube makes it harder to cough away debris that could irritate your lungs and cause an infection
 \item Medical staff members carefully measure the amount, type, speed, and force of the air the ventilator pushes into and pulls out of your lungs. Too much oxygen in the mix for too long can be bad for your lungs. If the force or amount of air is too much, or if your lungs are too weak, it can damage your lung tissue. Your doctor might call this ventilator-associated lung injury
 
 \end{itemize}
 \begin{figure}
\centering
\includegraphics[scale=0.9]{Ultrasound}
\caption{Ultrasound}
\end{figure}
 \section{Ultrasound}
an ultrasound is an imaging test that uses sound waves to create a picture (also known as a sonogram) of organs, tissues, and other structures inside the body. Unlike x-rays, ultrasounds don't use any radiation

\subsection{Ultrasound Parts}

\begin{itemize}
\item a transducer
\item probe; the processing unit
\item including the controls
\item the display
\end{itemize}
\subsection{Working}
 In an ultrasound exam, a transducer both sends the sound waves and records the echoing (returning) waves. When the transducer is pressed against the skin, it sends small pulses of inaudible, high-frequency sound waves into the body.
\subsection{Uses}
\begin{itemize}
\item View the uterus and ovaries during pregnancy and monitor the developing baby's health
\item Diagnose gallbladder disease
\item  Evaluate blood flow
\item  Examine a breast lump
\item  Check your thyroid gland
\item Detect genital and prostate problems
\end{itemize}
\subsection{Benefits}
\begin{itemize}
\item They are generally painless and do not require needles, injections, or incisions.
\item Patients aren't exposed to ionizing radiation, making the procedure safer than diagnostic techniques such as X-rays and CT scans. In fact, there are no known harmful effects when used as directed by your health care provider.
\item Ultrasound captures images of soft tissues that don't show up well on X-rays
\item Ultrasounds are widely accessible and less expensive than other methods.
\end{itemize}
\subsection{Risks}
While ultrasound is generally considered to be safe with very low risks, the risks may increase with unnecessary prolonged exposure to ultrasound energy, or when untrained users operate the device.



\end{document}
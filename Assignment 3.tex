\documentclass[12pt]{article}

\begin{document}

\clearpage

\section*{ASSIGNMENT-3}

\subsection*{}
\begin{itemize}
\item Name       :  Medha Kumari
\item Roll no.   :  21111030
\end{itemize}
\section*{The Future of Healthcare}
As the baby boomer generation approaches retirement, thus qualifying for Medicare, healthcare spending by federal, state, and local governments is projected to increase. Assuming the government continues to subsidize Marketplace premiums for lower-income populations, this increased government healthcare spending will greatly affect the entire healthcare system in the U.S. Although Medicaid spending growth decelerated in 2016 due to reduced enrollment, spending is expected to accelerate at an average rate of 7.1 percent per year in 2018 and 2019 due to the aging baby boomer generation.
\subsection*{A Shift in Healthcare Providers}
Along with policy and technological changes, the people who provide healthcare are also changing. Providers are an important part of the healthcare system and any changes to their education, satisfaction or demographics are likely to affect how patients receive care.

Future healthcare providers are also more likely to focus their education on business than ever before. A large-scale analysis of Harvard Business School’s physician graduates indicates substantial growth in the number of physicians pursuing M.B.A. degrees in the last decade. This growth may result in more private practices and healthcare administrators.
\subsubsection*{Demographics}
In recent years, the demographics of the medical profession have shifted. Women currently make up the majority of healthcare providers in certain specialties, including pediatrics and obstetrics and gynecology. Nearly one-third of all practicing physicians are women.Based on statistics, we can assume more women may enter the medical profession in the coming years.

This shift in demographics to include more women in healthcare supports diversity in the industry and represents overall population diversity.
\subsubsection*{Competence}
The prevalence of malpractice lawsuits is one way to evaluate the competence of healthcare providers. The amount of malpractice claims in the U.S. has steadily declined since 2004. As the trend of declining malpractice lawsuits continues, it may indicate that provider competence and patient care will continue to improve.
\subsubsection*{Satisfaction}
Job satisfaction is one area that must improve. According to Medscape’s 2015 Physician Compensation Report, 64 percent of doctors would choose medicine as a career if they could do it again, but only 45 percent would choose the same area of specialty. Nurses report higher overall career satisfaction than doctors, based on results of the latest Survey of Registered Nurses conducted by AMN Healthcare and compared to the Physician Compensation Report. Nine out of 10 nurses who participated in the survey said they were satisfied with their career choice. However, one out of every three nurses is unhappy with their current job. It is difficult to say whether job satisfaction will increase in the coming years, but continued technological advancements designed to streamline the healthcare process offer hope to those who may be frustrated with the complexity of their jobs.
\subsection*{Evolving Needs of Patients}
Demands on healthcare change due to various reasons, including the needs of patients. Every year, new cures and treatments help manage common diseases. Each such development affects the entire healthcare system as much as it has a positive impact on patients. As illnesses become more common, our healthcare system must adapt to treat them. Patient care needs will also evolve as the population ages and relies more heavily on resources such as Medicare and Medicaid. Patient empowerment is expected to increase with advances in technology.
\subsubsection*{Illness Trends}
The bubonic plague is a good example of a disease that can drastically change the healthcare system by quickly shifting all resources to handle an epidemic. In the Middle Ages, the Black Death spread so quickly across Europe that it is responsible for an estimated 75 million deaths. It may be surprising that the bubonic plague still circulates today. In fact, according to Center for Disease Control data, there were 11 cases and three deaths in the U.S. within five months in 2015.

Although the bubonic plague is not near the threat it once was, other diseases and conditions of concern are on the rise. The following seven conditions are on the rise and can be expected to have an impact on healthcare in the near future:
\begin{itemize}
\item Sexually Transmitted Infections: Chlamydia and gonorrhea rates have increased, and syphilis rates rose by 15.1 percent from 2013 to 2014.
\item Obesity: Obesity continues to be an issue in the U.S. with 78.6 million adults and 12.7 million children affected. Obesity rates have increased by 17 percent in the past five years.
\item Autism: For every 100,000 people, 1,470 are diagnosed with autism. This number continues to rise annually. Recent increases may be due to awareness as doctors become more familiar with the symptoms of autism.
\item E. Coli: Within 10 years, cases of E. coli have increased by 472 percent. Many E. coli cases are the result of food contamination.
\item Liver Cancer: Incidences of liver cancer have increased by 47 percent in a recent 10-year timeframe.
\item Kidney Cancer: Healthcare practitioners have treated 18.6 percent more cases of kidney cancer in the past 10 years than in previous years.
\item Whooping Cough: The 10-year increase for whooping cough is nearly 146 percent. This may be due in part to parents opting out of whooping cough vaccinations.
\end{itemize}
The healthcare industry has identified these previous conditions, preparing to handle further increases with supplies and resources. However, a new threat is always possible. If something similar to the Ebola virus spread across the country, this would have a drastic impact on patient care and healthcare facilities.
\subsubsection*{Population Shift}
The current baby boomer generation, which initially consisted of 76 million people born between 1946 and 1964, will be coming to retirement age and will increase federal spending on Medicare and Medicaid by an average of 5.9 percent in 2018 and 2019.
\subsubsection*{Advances in Technology}
Healthcare technology trends focus heavily on patient empowerment. The introduction of wearable biometric devices that provide patients with information about their own health and telemedicine apps allow patients to easily access care no matter where they live. With new technologies focused on monitoring, research, and healthcare availability, patients will be able to take a more active role in their care.
\subsection*{Conclusion}
From policy to patients and everything in-between, the healthcare industry is constantly evolving. Aging populations, technological advancements, and illness trends all have an impact on where healthcare is headed. Since it is crucial to pay attention to shifts in society to understand where healthcare is headed, consider dedicating time each day to reading recommended industry literature that you will find in our list of 25 books for every healthcare professional.
\end{document}


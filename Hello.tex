\documentclass[12pt]{article}
\usepackage{graphicx}
\graphicspath{{images}}
\begin{document}
\tableofcontents
\clearpage

\section{CARBON DIOXIDE SENSOR}


A carbon dioxide sensor measures gaseous carbon dioxide levels by monitoring the amount of infrared (IR) radiation absorbed by carbon dioxide molecules. Measuring carbon dioxide is critical in monitoring many industrial processes and indoor air quality

\begin{figure}
\centering
\includegraphics[scale=0.7]{SENSOR.jpeg}
\caption{CARBON DIOXIDE SENSOR}
\end{figure}

\section*{TYPES OF CO2 SENSORS}
The following are two major types of carbon dioxide sensors:

\subsection{Nondispersive Infrared (NDIR) Carbon Dioxide Sensors}
NDIR sensors are spectroscopic sensors that detect carbon dioxide in a gaseous environment by its characteristic absorption. It includes an infrared detector, an interference filter, a light tube and an infrared source. The gas is either pumped or diffused into the light tube, and the electronics measure the absorption of the wavelength of light.
\subsection{Chemical Carbon Dioxide Sensors}
Chemical carbon dioxide gas sensors consist of sensitive layers based on polymer- or heteropolysiloxane. They have very low energy consumption and can be reduced in size to fit into microelectronic-based systems.


\subsection{Working Principle}

The carbon dioxide gas sensor measures gaseous carbon dioxide levels by detecting the quantity of IR radiation absorbed by carbon dioxide molecules. The sensor employs a hot metal filament that acts as an IR source to generate IR radiation.

The sensor measures the IR radiation absorbed in the narrow band at 4260 nm. The greater the absorbing gas concentration in the sampling tube, the lesser the amount of radiation from the source.

As a result of this increase in temperature, a voltage is generated, amplified and read by an interface system. Meanwhile, the carbon dioxide gas diffuses via the sensor tube by the eight vent holes.

\subsection{Applications}
Some of the major applications of carbon dioxide sensors include the following:

\begin{itemize}
\item It can be used for HVAC applications to monitor the quality of air.
\item It is used to monitor fermentation, aerobic respiration, photosynthesis and other carbon dioxide consuming or producing processes.
\item It can be used for indoor human occupancy counting.

\end{itemize}

\clearpage


\section{SPHYGMOMANOMETER}
\begin{figure}
\centering
\includegraphics[scale=0.2]{Sphygmomanometer.jpg}
\caption{Sphygnomanometer}
\end{figure}


A sphygmomanometer also known as a blood pressure monitor, or blood pressure gauge, is a device used to measure blood pressure, composed of an inflatable cuff to collapse and then release the artery under the cuff in a controlled manner, and a mercury or aneroid manometer to measure the pressure. 

\subsection*{TYPES}
\subsection{Manual}
Manual meters are best used by trained practitioners, and, while it is possible to obtain a basic reading through palpation alone, this yields only the systolic pressure.

Mercury sphygmomanometers are considered the gold standard. They indicate pressure with a column of mercury, which does not require recalibration Because of their accuracy, they are often used in clinical trials of drugs and in clinical evaluations of high-risk patients, including pregnant women. A frequently used wall mounted mercury sphygmomanometer is also known as a Baumanometer.


\subsection{Digital}
Digital meters employ oscillometric measurements and electronic calculations rather than auscultation. They may use manual or automatic inflation, but both types are electronic, easy to operate without training, and can be used in noisy environments. 
They measure mean blood pressure and pulse rate, while systolic and diastolic pressures are obtained less accurately than with manual meters,and calibration is also a concern.

\subsection*{Impact of Sphygmomanometer}
\subsubsection{Advantages}
The benefits can either be functional or economical.
One of the outstanding benefits is its use to measure the blood pressure. Another common function benefit of sphygmomanometers is that most of them are portable and can easily be put using hand and be carried from one place to another.
It means this device type is effective and can be used anywhere across the globe. Aneroid also has various functional and economical benefits available to the users. For instance, aneroid sphygmomanometer has an inbuilt stethoscope thus the user does not need to purchase one separately.
\subsubsection{Disadvantages}
Nevertheless, sphygmomanometers also have their negative impacts. mercury sphygmomanometer has negative impact on the environment and human beings as identified by the World Health Organization and United Nations Environmental Programme. Mercury used in this device is most lethal pollutant of the environment.
This device is that it is fragile and can easily break when not handled well.

\subsection{Conclusion}
To conclude, high pressure has been described as one of the major risk factors for other diseases. Imbalance of this blood pressure can lead to other diseases such as stroke, renal failure and heart failure among others. Therefore, discovery sphygmomanometer by Karl Samuel Ritter Von Basch has been a major breakthrough in clinical practice.

\section{ARTHOSCOPE}



\begin{figure}
\centering
\includegraphics[scale=0.5]{Arthoscope.jpg}
\caption{Arthoscope}
\end{figure}

A thin flexible fiberoptic scope which is introduced into a joint space through a small incision in order to carry out diagnostic and treatment procedures within the joint. An arthroscope is about the diameter of a drinking straw. It is fitted with a miniature camera, a light source and precision tools at the end of flexible tubes.


\subsection{History}
The arthroscope was invented by the Japanese physician Masaki Watanabe, in the early 1960s to permit orthopaedic surgery in a relatively noninvasive way. Watanabe based his invention on the cystoscope, an instrument used to examine the inside of the bladder. The first arthroscope had a tiny camera lens mounted on a flexible tube, which allowed a surgeon to peer into the interior of joints through a small incision. Over the years, the arthroscope has been refined but not fundamentally changed.

\subsection{Principle of Operation}

Once the arthroscope is inserted, the lens system magnifies
the inside of the joint by gathering light and focusing that light
to form a real image. Surrounding optical fibers transmit light
to the far end of the scope to aid visibility. The field of view
is proportional to an arthroscope’s diameter and a variety of
diameters are offered by manufacturers. An instrument channel
can allow tools to be passed down the optical line of the
arthroscope.

\subsection{Uses}
An arthroscope can be used not only for diagnostic procedures but a wide range of surgical repairs, such as debridement, or cleaning, of a joint to remove bits of torn cartilage, ligament reconstruction, and synovectomy (removal of the joint lining). All are done without a major, invasive operation, and, since arthroscopy requires only tiny incisions, many procedures can be done on an outpatient basis with local anesthetic.

\clearpage


 
\section{ KERATOMETER }
\begin{figure}
\centering
\includegraphics[scale=0.4]{Keratometer.jpg}
\caption{Keratometer}
\end{figure}
 
A keratometer (sometimes referred to as an ophthalmometer) is a device used for measuring the curvature of the cornea’s frontal surface. In 1851, German physiologist Hermann von Helmholtz built the first keratometer, although a similar instrument was built by Jesse Ramsden and Everard Home in 1796. This diagnosing instrument helps in evaluating the degree and axis of astigmatism.
 \subsection{Uses}
 A keratometer provides the following information:
 \begin{itemize}
\item Evaluating degree of astigmatism .
\item Fitting of eyeglasses and contact lenses.
\item Analyzing patients having keratoconus.
\item Determining intraocular lens power for patients suffering from cataracts.
\end{itemize}
\subsection{Principle of Keratometer}
The main purpose of a keratometer is to find the optical refracting power of the cornea (that is a convex refracting surface). When an object of known size is placed at a known distance from the corneal surface in such a way that the size of the reflected image of the object is measured by a measuring telescope.

R = 2 d (I/O)

R: Radius of curvature of the cornea in meters.
d: Distance of the object from the cornea.
I: Size of the image.
O: Size of the object.


Now, the refractive power of the cornea can be given by the formula

D = (n-1)/R
D: Dioptric refracting power of the corneal surface.
n: refractive index of the instrument (n= 1.3375 generally)
A keratometer displays the curvature of the cornea in terms of diopters of power or in terms of millimeters and diopters. If the readings are shown in terms of millimeters then the dioptric power can be derived by the equation shown above.

\subsection{Types}
\subsubsection{Javal-Schiotz Keratometer}
The Javal-Schiotz keratometer is based on the Javal-Schiotz principle that works with a fixed image, a doubling size, and an adjustable object size for obtaining the curvature of the corneal surface. It is a two-position instrument and uses two self-illuminated objects. One of the objects is a red square and the other is green staircase-like design.
\subsubsection{Bausch and Lomb Keratometer}
The Bausch and Lomb Keratometer is based on the Bausch and Lomb principle that works with a fixed object instead of a fixed image. This is a one-position keratometer and the image size can be varied.

\subsection{Conclusion}
Keratometry requires precise measurements as eye operations are critical and often irreversible. So, errors in measurements can cause huge issues. While automated keratometers are useful and comparatively easy to use for beginners, manual keratometers provide a more accurate reading. An experienced professional generally prefer using the manual instrument for fine adjustments. However, nowadays updated automated instruments are comparable to manual design. It is very important to be completely sure about the readings before one orders an eye lens. Both the patient and the doctor should be satisfied with the reading values.
 
 \section{NEBULISER}
 
 \begin{figure}
\centering
\includegraphics[scale=0.6]{Nebuliser.jpeg}
\caption{Nebuliser}
\end{figure}

There are several ways to deliver medicine to the lungs, such as inhalers. Nebulizers are another type of drug delivery device that turns the medicine into a fine mist that is breathed in. Nebulizers are used to treat a variety of lung conditions, such as wheezing and chest tightness.

\subsection{Uses and Applications}
Nebulizers can be used to relieve symptoms of various lung conditions by administering both quick-relief medicines and long-term control medicines. Most inhaled medications relieve symptoms such as wheeziness, breathlessness, and tightness in the chest. They can also prevent or slow the accumulation of phlegm and mucus.

Nebulizers can also be used in less severe cases at home by individual users. Inhalers are less effective than nebulizers, so nebulizers can be used if the condition it is treating is severe.

 Some other health conditions, such as arthritis, can prevent patients from using inhalers. Furthermore, nebulizers are used for young children and babies. Nebulizers can deliver drugs such as bronchodilators to open the airways, hypertonic saline solutions to loosen up mucus, and antibiotics to treat or prevent infections.
 
 \subsection{Nebulizers and COVID-19}
 
 COVID-19 is a serious respiratory disease that is currently spreading around the world. As a respiratory disease that can severely affect the lungs, nebulizers have been used in self-medicating and professional settings to manage lung conditions alongside COVID-19. However, there are some concerns about this.
 
 Furthermore, there is the possibility that many aerosols are generated that can spread over longer distances than would otherwise naturally occur when they are exhaled from the device. Therefore, while the nebulizer device itself may not spread disease, the effects of droplet formation can lead to transmission of COVID-19 to bystander hosts.
There is also some concern over the immunosuppressive qualities of many inhaled corticosteroids. This could contribute to recommending against the use of nebulizers for patients with COVID-19 or those in proximity to it. 



\end{document}
